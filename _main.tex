\PassOptionsToPackage{unicode=true}{hyperref} % options for packages loaded elsewhere
\PassOptionsToPackage{hyphens}{url}
%
\documentclass[]{book}
\usepackage{lmodern}
\usepackage{amssymb,amsmath}
\usepackage{ifxetex,ifluatex}
\usepackage{fixltx2e} % provides \textsubscript
\ifnum 0\ifxetex 1\fi\ifluatex 1\fi=0 % if pdftex
  \usepackage[T1]{fontenc}
  \usepackage[utf8]{inputenc}
  \usepackage{textcomp} % provides euro and other symbols
\else % if luatex or xelatex
  \usepackage{unicode-math}
  \defaultfontfeatures{Ligatures=TeX,Scale=MatchLowercase}
\fi
% use upquote if available, for straight quotes in verbatim environments
\IfFileExists{upquote.sty}{\usepackage{upquote}}{}
% use microtype if available
\IfFileExists{microtype.sty}{%
\usepackage[]{microtype}
\UseMicrotypeSet[protrusion]{basicmath} % disable protrusion for tt fonts
}{}
\IfFileExists{parskip.sty}{%
\usepackage{parskip}
}{% else
\setlength{\parindent}{0pt}
\setlength{\parskip}{6pt plus 2pt minus 1pt}
}
\usepackage{hyperref}
\hypersetup{
            pdftitle={Diagnostics Supplemental Material},
            pdfauthor={Jose Guadalupe Hernandez},
            pdfborder={0 0 0},
            breaklinks=true}
\urlstyle{same}  % don't use monospace font for urls
\usepackage{color}
\usepackage{fancyvrb}
\newcommand{\VerbBar}{|}
\newcommand{\VERB}{\Verb[commandchars=\\\{\}]}
\DefineVerbatimEnvironment{Highlighting}{Verbatim}{commandchars=\\\{\}}
% Add ',fontsize=\small' for more characters per line
\usepackage{framed}
\definecolor{shadecolor}{RGB}{248,248,248}
\newenvironment{Shaded}{\begin{snugshade}}{\end{snugshade}}
\newcommand{\AlertTok}[1]{\textcolor[rgb]{0.94,0.16,0.16}{#1}}
\newcommand{\AnnotationTok}[1]{\textcolor[rgb]{0.56,0.35,0.01}{\textbf{\textit{#1}}}}
\newcommand{\AttributeTok}[1]{\textcolor[rgb]{0.77,0.63,0.00}{#1}}
\newcommand{\BaseNTok}[1]{\textcolor[rgb]{0.00,0.00,0.81}{#1}}
\newcommand{\BuiltInTok}[1]{#1}
\newcommand{\CharTok}[1]{\textcolor[rgb]{0.31,0.60,0.02}{#1}}
\newcommand{\CommentTok}[1]{\textcolor[rgb]{0.56,0.35,0.01}{\textit{#1}}}
\newcommand{\CommentVarTok}[1]{\textcolor[rgb]{0.56,0.35,0.01}{\textbf{\textit{#1}}}}
\newcommand{\ConstantTok}[1]{\textcolor[rgb]{0.00,0.00,0.00}{#1}}
\newcommand{\ControlFlowTok}[1]{\textcolor[rgb]{0.13,0.29,0.53}{\textbf{#1}}}
\newcommand{\DataTypeTok}[1]{\textcolor[rgb]{0.13,0.29,0.53}{#1}}
\newcommand{\DecValTok}[1]{\textcolor[rgb]{0.00,0.00,0.81}{#1}}
\newcommand{\DocumentationTok}[1]{\textcolor[rgb]{0.56,0.35,0.01}{\textbf{\textit{#1}}}}
\newcommand{\ErrorTok}[1]{\textcolor[rgb]{0.64,0.00,0.00}{\textbf{#1}}}
\newcommand{\ExtensionTok}[1]{#1}
\newcommand{\FloatTok}[1]{\textcolor[rgb]{0.00,0.00,0.81}{#1}}
\newcommand{\FunctionTok}[1]{\textcolor[rgb]{0.00,0.00,0.00}{#1}}
\newcommand{\ImportTok}[1]{#1}
\newcommand{\InformationTok}[1]{\textcolor[rgb]{0.56,0.35,0.01}{\textbf{\textit{#1}}}}
\newcommand{\KeywordTok}[1]{\textcolor[rgb]{0.13,0.29,0.53}{\textbf{#1}}}
\newcommand{\NormalTok}[1]{#1}
\newcommand{\OperatorTok}[1]{\textcolor[rgb]{0.81,0.36,0.00}{\textbf{#1}}}
\newcommand{\OtherTok}[1]{\textcolor[rgb]{0.56,0.35,0.01}{#1}}
\newcommand{\PreprocessorTok}[1]{\textcolor[rgb]{0.56,0.35,0.01}{\textit{#1}}}
\newcommand{\RegionMarkerTok}[1]{#1}
\newcommand{\SpecialCharTok}[1]{\textcolor[rgb]{0.00,0.00,0.00}{#1}}
\newcommand{\SpecialStringTok}[1]{\textcolor[rgb]{0.31,0.60,0.02}{#1}}
\newcommand{\StringTok}[1]{\textcolor[rgb]{0.31,0.60,0.02}{#1}}
\newcommand{\VariableTok}[1]{\textcolor[rgb]{0.00,0.00,0.00}{#1}}
\newcommand{\VerbatimStringTok}[1]{\textcolor[rgb]{0.31,0.60,0.02}{#1}}
\newcommand{\WarningTok}[1]{\textcolor[rgb]{0.56,0.35,0.01}{\textbf{\textit{#1}}}}
\usepackage{longtable,booktabs}
% Fix footnotes in tables (requires footnote package)
\IfFileExists{footnote.sty}{\usepackage{footnote}\makesavenoteenv{longtable}}{}
\usepackage{graphicx,grffile}
\makeatletter
\def\maxwidth{\ifdim\Gin@nat@width>\linewidth\linewidth\else\Gin@nat@width\fi}
\def\maxheight{\ifdim\Gin@nat@height>\textheight\textheight\else\Gin@nat@height\fi}
\makeatother
% Scale images if necessary, so that they will not overflow the page
% margins by default, and it is still possible to overwrite the defaults
% using explicit options in \includegraphics[width, height, ...]{}
\setkeys{Gin}{width=\maxwidth,height=\maxheight,keepaspectratio}
\setlength{\emergencystretch}{3em}  % prevent overfull lines
\providecommand{\tightlist}{%
  \setlength{\itemsep}{0pt}\setlength{\parskip}{0pt}}
\setcounter{secnumdepth}{5}
% Redefines (sub)paragraphs to behave more like sections
\ifx\paragraph\undefined\else
\let\oldparagraph\paragraph
\renewcommand{\paragraph}[1]{\oldparagraph{#1}\mbox{}}
\fi
\ifx\subparagraph\undefined\else
\let\oldsubparagraph\subparagraph
\renewcommand{\subparagraph}[1]{\oldsubparagraph{#1}\mbox{}}
\fi

% set default figure placement to htbp
\makeatletter
\def\fps@figure{htbp}
\makeatother

\usepackage[]{natbib}
\bibliographystyle{apalike}

\title{Diagnostics Supplemental Material}
\author{Jose Guadalupe Hernandez}
\date{2022-08-03}

\begin{document}
\maketitle

{
\setcounter{tocdepth}{1}
\tableofcontents
}
\hypertarget{introduction}{%
\chapter{Introduction}\label{introduction}}

This is the supplemental material associated with our 2022 ECJ contribution entitled, \emph{A suite of diagnostic metrics for characterizing selection schemes}.
Preprint \href{https://arxiv.org/pdf/2204.13839.pdf}{here}.

\hypertarget{about-our-supplemental-material}{%
\section{About our supplemental material}\label{about-our-supplemental-material}}

This supplemental material is hosted on \href{https://github.com}{GitHub} using GitHub pages.
The source code and configuration files used to generate this supplemental material can be found in \href{https://github.com/jgh9094/ECJ-2022-suite-of-diagnostics-for-selection-schemes}{this GitHub repository}.
We compiled our data analyses and supplemental documentation into this nifty web-accessible book using \href{https://bookdown.org/}{bookdown}.

Our supplemental material includes the following paper figures and statistics:

\begin{itemize}
\tightlist
\item
  Exploitation rate results (Section \ref{exploitation-rate-results})
\item
  Ordered exploitation results (Section \ref{ordered-exploitation-results})
\item
  Contradictory objectives results (Section \ref{contradictory-objectives-results})
\item
  Multi-path exploration results (Section \ref{multi-path-exploration-results})
\end{itemize}

Additionally, our supplemental material includes the results from parameter tuning selection schemes:

\begin{itemize}
\tightlist
\item
  Truncation selection (Section \ref{truncation-selection})
\item
  Tournament selection sharing (Section \ref{tournament-selection})
\item
  Genotypic fitness sharing (Section \ref{genotypic-fitness-sharing})
\item
  Phenotypic fitness sharing (Section \ref{phenotypic-fitness-sharing})
\item
  Nondominated sorting (Section \ref{nondominated-sorting})
\item
  Novelty search (Section \ref{novelty-search})
\end{itemize}

\hypertarget{contributing-authors}{%
\section{Contributing authors}\label{contributing-authors}}

\begin{itemize}
\tightlist
\item
  \href{https://jgh9094.github.io/}{Jose Guadalupe Hernandez}
\item
  \href{https://lalejini.com}{Alexander Lalejini}
\item
  \href{http://ofria.com}{Charles Ofria}
\end{itemize}

\hypertarget{research-overview}{%
\section{Research overview}\label{research-overview}}

Abstract:

Evolutionary algorithms are effective general-purpose techniques for solving optimization problems.
Typically, evolutionary algorithms consist of multiple interacting components, where each component influences an algorithm's problem-solving abilities.
Understanding how each component of an evolutionary algorithm influences its problem-solving success improves our ability to target particular problem domains.
Benchmark suites provide insights into an evolutionary algorithm's problem-solving capabilities, but benchmarking problems often have complex search space topologies, making it difficult to isolate and test an algorithm's strengths and weaknesses.
Our work focuses on diagnosing selection schemes, which identity individuals to contribute genetic material to the next generation, thus driving an evolutionary algorithm's search strategy.
We introduce four diagnostics for empirically testing the strengths and weaknesses of selection schemes: the exploitation rate diagnostic, ordered exploitation rate diagnostic, contradictory objectives diagnostic, and the multi-path exploration diagnostic.
Each diagnostic is a handcrafted search space designed to isolate and measure the relative exploitation and exploration characteristics of selection schemes.
Here, we use our diagnostics to evaluate six population selection methods: truncation selection, tournament selection, fitness sharing, lexicase selection, nondominated sorting, and novelty search.
Expectedly, tournament and truncation selection excelled at gradient exploitation but poorly explored search spaces, and novelty search excelled at exploration but failed to exploit fitness gradients.
Fitness sharing performed poorly across all diagnostics, suggesting poor overall exploitation and exploration abilities.
Nondominated sorting was best for maintaining diverse populations comprised of individuals inhabiting multiple optima, but struggled to effectively exploit fitness gradients.
Lexicase selection balanced search space exploration with exploitation, generally performing well across diagnostics.
Our work demonstrates the value of diagnostic search spaces for building a deeper understanding of selection schemes, which can then be used to improve or develop new selection methods.

\hypertarget{experimental-setup}{%
\section{Experimental setup}\label{experimental-setup}}

Setting up required variables variables.

\begin{Shaded}
\begin{Highlighting}[]
\CommentTok{# source("./DataTools/Vizualizor/setup.R")}
\end{Highlighting}
\end{Shaded}

These analyses were conducted in the following computing environment:

\begin{Shaded}
\begin{Highlighting}[]
\KeywordTok{print}\NormalTok{(version)}
\end{Highlighting}
\end{Shaded}

\begin{verbatim}
##                _                           
## platform       x86_64-pc-linux-gnu         
## arch           x86_64                      
## os             linux-gnu                   
## system         x86_64, linux-gnu           
## status                                     
## major          4                           
## minor          2.1                         
## year           2022                        
## month          06                          
## day            23                          
## svn rev        82513                       
## language       R                           
## version.string R version 4.2.1 (2022-06-23)
## nickname       Funny-Looking Kid
\end{verbatim}

\bibliography{book.bib,packages.bib}

\end{document}
